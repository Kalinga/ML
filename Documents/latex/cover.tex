\title{Deep Learning: Pedestrian trajectory detection}

\author{Kalinga Bhusan Ray}

\department{Fakult{\"a}t f{\"u}r Informatik und Automatisierung,\\Fachgebiet Simulation und Optimale Prozesse}

\degree{Master of Science (M. Sc.) in  Research in Computer \& Systems Engineering}

%\newcommand{\artderausarbeitung}{Bachelorarbeit}
%\newcommand{\namedesautors}{Max Mustermann}
%\newcommand{\themaderarbeit}{Anfertigung einer Ausarbeitung mit \LaTeX}

% PDF Metadaten definieren
%\hypersetup{
%   pdftitle={\themaderarbeit},
%   pdfsubject={\artderausarbeitung},
%   pdfauthor={\namedesautors},
%   pdfkeywords={\artderausarbeitung; TU-Ilmenau; Kommunikationsnetze;}}
	
% As of the 2007-08 academic year, valid degree months are September, 
% February, or June.  The default is June.
\dateofbirth{Date and Place of birth: 02. June 1983, India}
\matrikel{Matriculation No: 59504}

\degreemonth{Oct}
\degreeyear{2019}
\thesisdate{Oct 15, 2019}

%% By default, the thesis will be copyrighted to MIT.  If you need to copyright
%% the thesis to yourself, just specify the `vi' documentclass option.  If for
%% some reason you want to exactly specify the copyright notice text, you can
%% use the \copyrightnoticetext command.  
%\copyrightnoticetext{\copyright IBM, 1990.  Do not open till Xmas.}

% If there is more than one supervisor, use the \supervisor command
% once for each.
\supervisor{Prof. Dr.-Ing. habil. Pu Li}{University Professor}

% This is the department committee chairman, not the thesis committee
% chairman.  You should replace this with your Department's Committee
% Chairman.
\chairman{Qais Mohammed Ali Yousef}{Academic supervisor}

% Make the titlepage based on the above information.  If you need
% something special and can't use the standard form, you can specify
% the exact text of the titlepage yourself.  Put it in a titlepage
% environment and leave blank lines where you want vertical space.
% The spaces will be adjusted to fill the entire page.  The dotted
% lines for the signatures are made with the \signature command.
\maketitle

% The abstractpage environment sets up everything on the page except
% the text itself.  The title and other header material are put at the
% top of the page, and the supervisors are listed at the bottom.  A
% new page is begun both before and after.  Of course, an abstract may
% be more than one page itself.  If you need more control over the
% format of the page, you can use the abstract environment, which puts
% the word "Abstract" at the beginning and single spaces its text.

%% You can either \input (*not* \include) your abstract file, or you can put
%% the text of the abstract directly between the \begin{abstractpage} and
%% \end{abstractpage} commands.

% First copy: start a new page, and save the page number.
%\cleardoublepage

% Uncomment the next line if you do NOT want a page number on your
% abstract and acknowledgments pages.
\pagestyle{empty}
%\setcounter{savepage}{\thepage}
\begin{abstractpage}
% $Log: abstract.tex,v $
% Revision 1.1  93/05/14  14:56:25  starflt
% Initial revision
% 
% Revision 1.1  90/05/04  10:41:01  lwvanels
% Initial revision
% 
%
%% The text of your abstract and nothing else (other than comments) goes here.
%% It will be single-spaced and the rest of the text that is supposed to go on
%% the abstract page will be generated by the abstractpage environment.  This
%% file should be \input (not \include 'd) from cover.tex.
\begin{flushleft}
In the recent times, Deep Learning plays a significant role in the computer vision
related tasks and Deep learning based algorithms such as 'Convolutional Neural Networks'
(CNN) demonstrated and proven to be outperforming other state-of-art models when 
deployed in object detection tasks. In the recent past many researchers shown great 
deals towards image classification and object detection problems. Also there exists several 
benchmarks in the context of classification and objects in an image detection. In 
comparison usage of deep learning to solve problems that involves prediction of 
user behavior from a monitored video is much less investigated.
\vskip 1\baselineskip
\par
During my Master Thesis, various aspects related to image classification, object 
in image detection and object tracking in a video and prediction of user behavior 
have been studied with a particular focus on Pedestrian trajectory prediction. 
A new prediction model is proposed and the results for pedestrian trajectory 
task are presented.
\vskip 1\baselineskip
\textit{Keywords: image classification, object detection, object tracking, trajectory prediction}
\end{flushleft}
\end{abstractpage}

% Additional copy: start a new page, and reset the page number.  This way,
% the second copy of the abstract is not counted as separate pages.
% Uncomment the next 6 lines if you need two copies of the abstract
% page.
% \setcounter{page}{\thesavepage}
% \begin{abstractpage}
% % $Log: abstract.tex,v $
% Revision 1.1  93/05/14  14:56:25  starflt
% Initial revision
% 
% Revision 1.1  90/05/04  10:41:01  lwvanels
% Initial revision
% 
%
%% The text of your abstract and nothing else (other than comments) goes here.
%% It will be single-spaced and the rest of the text that is supposed to go on
%% the abstract page will be generated by the abstractpage environment.  This
%% file should be \input (not \include 'd) from cover.tex.
\begin{flushleft}
In the recent times, Deep Learning plays a significant role in the computer vision
related tasks and Deep learning based algorithms such as 'Convolutional Neural Networks'
(CNN) demonstrated and proven to be outperforming other state-of-art models when 
deployed in object detection tasks. In the recent past many researchers shown great 
deals towards image classification and object detection problems. Also there exists several 
benchmarks in the context of classification and objects in an image detection. In 
comparison usage of deep learning to solve problems that involves prediction of 
user behavior from a monitored video is much less investigated.
\vskip 1\baselineskip
\par
During my Master Thesis, various aspects related to image classification, object 
in image detection and object tracking in a video and prediction of user behavior 
have been studied with a particular focus on Pedestrian trajectory prediction. 
A new prediction model is proposed and the results for pedestrian trajectory 
task are presented.
\vskip 1\baselineskip
\textit{Keywords: image classification, object detection, object tracking, trajectory prediction}
\end{flushleft}
% \end{abstractpage}

%\cleardoublepage

\section*{Acknowledgments}
%\begin{flushleft}
Countless thanks to my parents, in-laws, relatives, especially to my wife
for giving me unfailing support and continuous encouragement throughout my 
years of study and while writing this thesis. This accomplishment would not 
have been possible without their support and blessings. I would like to 
dedicate this work to my grandfather Sri Baishnab Charan Mohanty who 
believed in me all the time and blessed me. And sincere thanks to my sister Nibedita Roy for proof-reading my draft.

\vspace{1em}
\noindent My special thanks to my supervisor Qais Mohammed Ali Yousef for his valuable
inputs, support and inspirations. He believed in me and helped me a lot.
The door to his office was always accessible whenever I had a question 
or ran into some trouble. He always enabled this Thesis to be my own work 
but did not shy away from steering me in the right direction whenever deemed necessary.

\vspace{1em}
\noindent Many thanks to Prof. Dr. Pu Li, his Control Engineering course was a 
catalyst for me doing my research. Also many thanks to the SOP department for 
enabling a great working environment. Special thanks to Bj{\"o}rn T{\"o}pper, 
he helped me a lot in bringing the required system and administration.

\vspace{1em}
\noindent My sincere thanks to all the Professor and teaching staff of Research 
in Computer \& Systems Engineering (RCSE) course, it was a pleasant and lifetime experience studying this course at Technische Universit{\"a}t Ilmenau.

\vspace{1em}
\noindent Author

\vspace{1em}
\noindent Kalinga Bhusan Ray
%\end{flushleft}

%%%%%%%%%%%%%%%%%%%%%%%%%%%%%%%%%%%%%%%%%%%%%%%%%%%%%%%%%%%%%%%%%%%%%%
% -*-latex-*-
