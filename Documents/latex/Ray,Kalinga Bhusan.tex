% -*- Mode:TeX -*-

%% IMPORTANT: The official thesis specifications are available at:
%%            http://libraries.mit.edu/archives/thesis-specs/
%%
%%            Please verify your thesis' formatting and copyright
%%            assignment before submission.  If you notice any
%%            discrepancies between these templates and the 
%%            MIT Libraries' specs, please let us know
%%            by e-mailing thesis@mit.edu

%% The documentclass options along with the pagestyle can be used to generate
%% a technical report, a draft copy, or a regular thesis.  You may need to
%% re-specify the pagestyle after you \include  cover.tex.  For more
%% information, see the first few lines of mitthesis.cls. 

%\documentclass[12pt,vi,twoside]{mitthesis}
%%
%%  If you want your thesis copyright to you instead of MIT, use the
%%  ``vi'' option, as above.
%%
%\documentclass[12pt,twoside,leftblank]{mitthesis}
%%
%% If you want blank pages before new chapters to be labelled ``This
%% Page Intentionally Left Blank'', use the ``leftblank'' option, as
%% above. 

\documentclass[12pt,twoside]{mitthesis}
\usepackage{lgrind}
%% These have been added at the request of the MIT Libraries, because
%% some PDF conversions mess up the ligatures.  -LB, 1/22/2014
\usepackage{cmap}
\usepackage[T1]{fontenc}
\pagestyle{plain}

%% This bit allows you to either specify only the files which you wish to
%% process, or `all' to process all files which you \include.
%% Krishna Sethuraman (1990).

%\typein [\files]{Enter file names to process, (chap1,chap2 ...), or `all' to
%process all files:}
\def\all{all}
%\ifx\files\all \typeout{Including all files.} \else \typeout{Including only \files.} \includeonly{\files} \fi

\begin{document}


\title{Deep Learning: Pedestrian trajectory detection}

\author{Kalinga Bhusan Ray}

\department{Fakult{\"a}t f{\"u}r Informatik und Automatisierung}

\degree{Master of Science (M. Sc.) in  Research in Computer \& Systems Engineering}

%\newcommand{\artderausarbeitung}{Bachelorarbeit}
%\newcommand{\namedesautors}{Max Mustermann}
%\newcommand{\themaderarbeit}{Anfertigung einer Ausarbeitung mit \LaTeX}

% PDF Metadaten definieren
%\hypersetup{
%   pdftitle={\themaderarbeit},
%   pdfsubject={\artderausarbeitung},
%   pdfauthor={\namedesautors},
%   pdfkeywords={\artderausarbeitung; TU-Ilmenau; Kommunikationsnetze;}}
	
% As of the 2007-08 academic year, valid degree months are September, 
% February, or June.  The default is June.
\degreemonth{Oct}
\degreeyear{2019}
\thesisdate{Oct 15, 2019}

%% By default, the thesis will be copyrighted to MIT.  If you need to copyright
%% the thesis to yourself, just specify the `vi' documentclass option.  If for
%% some reason you want to exactly specify the copyright notice text, you can
%% use the \copyrightnoticetext command.  
%\copyrightnoticetext{\copyright IBM, 1990.  Do not open till Xmas.}

% If there is more than one supervisor, use the \supervisor command
% once for each.
\supervisor{Prof. Dr.-Ing. habil. Pu Li}{University Professor}

% This is the department committee chairman, not the thesis committee
% chairman.  You should replace this with your Department's Committee
% Chairman.
\chairman{Prof. Dr.-Ing. habil. Andreas Mitschele-Thiel}{Chairman, examination committee}

% Make the titlepage based on the above information.  If you need
% something special and can't use the standard form, you can specify
% the exact text of the titlepage yourself.  Put it in a titlepage
% environment and leave blank lines where you want vertical space.
% The spaces will be adjusted to fill the entire page.  The dotted
% lines for the signatures are made with the \signature command.
\maketitle

% The abstractpage environment sets up everything on the page except
% the text itself.  The title and other header material are put at the
% top of the page, and the supervisors are listed at the bottom.  A
% new page is begun both before and after.  Of course, an abstract may
% be more than one page itself.  If you need more control over the
% format of the page, you can use the abstract environment, which puts
% the word "Abstract" at the beginning and single spaces its text.

%% You can either \input (*not* \include) your abstract file, or you can put
%% the text of the abstract directly between the \begin{abstractpage} and
%% \end{abstractpage} commands.

% First copy: start a new page, and save the page number.
\cleardoublepage
% Uncomment the next line if you do NOT want a page number on your
% abstract and acknowledgments pages.
% \pagestyle{empty}
\setcounter{savepage}{\thepage}
\begin{abstractpage}
% $Log: abstract.tex,v $
% Revision 1.1  93/05/14  14:56:25  starflt
% Initial revision
% 
% Revision 1.1  90/05/04  10:41:01  lwvanels
% Initial revision
% 
%
%% The text of your abstract and nothing else (other than comments) goes here.
%% It will be single-spaced and the rest of the text that is supposed to go on
%% the abstract page will be generated by the abstractpage environment.  This
%% file should be \input (not \include 'd) from cover.tex.
\begin{flushleft}
In the recent times, Deep Learning plays a significant role in the computer vision
related tasks and Deep learning based algorithms such as 'Convolutional Neural Networks'
(CNN) demonstrated and proven to be outperforming other state-of-art models when 
deployed in object detection tasks. In the recent past many researchers shown great 
deals towards image classification and object detection problems. Also there exists several 
benchmarks in the context of classification and objects in an image detection. In 
comparison usage of deep learning to solve problems that involves prediction of 
user behavior from a monitored video is much less investigated.
\vskip 1\baselineskip
\par
During my Master Thesis, various aspects related to image classification, object 
in image detection and object tracking in a video and prediction of user behavior 
have been studied with a particular focus on Pedestrian trajectory prediction. 
A new prediction model is proposed and the results for pedestrian trajectory 
task are presented.
\vskip 1\baselineskip
\textit{Keywords: image classification, object detection, object tracking, trajectory prediction}
\end{flushleft}
\end{abstractpage}

% Additional copy: start a new page, and reset the page number.  This way,
% the second copy of the abstract is not counted as separate pages.
% Uncomment the next 6 lines if you need two copies of the abstract
% page.
% \setcounter{page}{\thesavepage}
% \begin{abstractpage}
% % $Log: abstract.tex,v $
% Revision 1.1  93/05/14  14:56:25  starflt
% Initial revision
% 
% Revision 1.1  90/05/04  10:41:01  lwvanels
% Initial revision
% 
%
%% The text of your abstract and nothing else (other than comments) goes here.
%% It will be single-spaced and the rest of the text that is supposed to go on
%% the abstract page will be generated by the abstractpage environment.  This
%% file should be \input (not \include 'd) from cover.tex.
\begin{flushleft}
In the recent times, Deep Learning plays a significant role in the computer vision
related tasks and Deep learning based algorithms such as 'Convolutional Neural Networks'
(CNN) demonstrated and proven to be outperforming other state-of-art models when 
deployed in object detection tasks. In the recent past many researchers shown great 
deals towards image classification and object detection problems. Also there exists several 
benchmarks in the context of classification and objects in an image detection. In 
comparison usage of deep learning to solve problems that involves prediction of 
user behavior from a monitored video is much less investigated.
\vskip 1\baselineskip
\par
During my Master Thesis, various aspects related to image classification, object 
in image detection and object tracking in a video and prediction of user behavior 
have been studied with a particular focus on Pedestrian trajectory prediction. 
A new prediction model is proposed and the results for pedestrian trajectory 
task are presented.
\vskip 1\baselineskip
\textit{Keywords: image classification, object detection, object tracking, trajectory prediction}
\end{flushleft}
% \end{abstractpage}

\cleardoublepage

\section*{Acknowledgments}
\begin{flushleft}
Countless thanks to my parents, in-laws and relatives, especially to my wife
for giving me unfailing support and continuous encouragement throughout my 
years of study and while writing this thesis. This accomplishment would not 
have been possible without them I would like to dedicate this work to my 
grandfather Sri Baishnab Charan Mohanty who believed in me all the times
and blessed me.
\vskip 1\baselineskip
%\par
My special thanks to my supervisor Qais Mohammed Ali Yousef for his valuable
inputs, support and inspirations. He believed in me and helped me a lot.
The door to his office was always accessible whenever I had a question 
or ran into some trouble. He always enabled this Thesis to be my own work 
but did not shy away from steering me in the right direction whenever deemed necessary.
\vskip 1\baselineskip
\par
Many thanks to Prof. Dr. Pu Li, his Control Engineering course was a 
catalyst for me doing my research. Also many thanks to SOP department for 
enabling a great working environment. Special thanks to Bj{\"o}rn T{\"o}pper, 
he helped me a lot in bringing the required system and administration.
\vskip 1\baselineskip
\par
My deeply thanks to all the Professor and teaching staffs of Research 
in Computer \& Systems Engineering (RCSE) course, it was a pleasant and life 
time experience studying this course at Technische Universit{\"a}t Ilmenau.
\vskip 1\baselineskip
\par
Author
Kalinga Bhusan Ray
\end{flushleft}

%%%%%%%%%%%%%%%%%%%%%%%%%%%%%%%%%%%%%%%%%%%%%%%%%%%%%%%%%%%%%%%%%%%%%%
% -*-latex-*-

% Some departments (e.g. 5) require an additional signature page.  See
% signature.tex for more information and uncomment the following line if
% applicable.
% % -*- Mode:TeX -*-
%
% Some departments (e.g. Chemistry) require an additional cover page
% with signatures of the thesis committee.  Please check with your
% thesis advisor or other appropriate person to determine if such a 
% page is required for your thesis.  
%
% If you choose not to use the "titlepage" environment, a \newpage
% commands, and several \vspace{\fill} commands may be necessary to
% achieve the required spacing.  The \signature command is defined in
% the "mitthesis" class
%
% The following sample appears courtesy of Ben Kaduk <kaduk@mit.edu> and
% was used in his June 2012 doctoral thesis in Chemistry. 

\begin{titlepage}
\begin{large}
This doctoral thesis has been examined by a Committee of the Department
of Chemistry as follows:

\signature{Professor Jianshu Cao}{Chairman, Thesis Committee \\
   Professor of Chemistry}

\signature{Professor Troy Van Voorhis}{Thesis Supervisor \\
   Associate Professor of Chemistry}

\signature{Professor Robert W. Field}{Member, Thesis Committee \\
   Haslam and Dewey Professor of Chemistry}
\end{large}
\end{titlepage}


\pagestyle{plain}
  % -*- Mode:TeX -*-
%% This file simply contains the commands that actually generate the table of
%% contents and lists of figures and tables.  You can omit any or all of
%% these files by simply taking out the appropriate command.  For more
%% information on these files, see appendix C.3.3 of the LaTeX manual. 
\pagenumbering{roman}
\setcounter{page}{1}
\tableofcontents
\newpage
\listoffigures
\newpage
\listoftables

%% This is an example first chapter.  You should put chapter/appendix that you
%% write into a separate file, and add a line \include{yourfilename} to
%% main.tex, where `yourfilename.tex' is the name of the chapter/appendix file.
%% You can process specific files by typing their names in at the 
%% \files=
%% prompt when you run the file main.tex through LaTeX.
\chapter{Introduction}
\section{Motivation}
Deep Learning has made some significant impacts in the areas of image classification, 
object detection, speech recognition, text-to-speech generation, machine translation,
online recommender system, medical diagnosis and few more. With availability of high 
computing power such as high speed CPU and high bandwidth GPU, large amount of 
data from which actionable insight is expected and nice and powerful ecosystem of  
tools such as TensorFlow and PyTorch; many researchers in the scientific community 
starting from neurologist to computer scientist are enticed towards research in the area 
of Artificial Intelligence. Several interesting findings, new algorithms, record 
breaking benchmarking results while applying Deep Learning algorithms has taken 
AI to a different level.

\vspace{1em}
\noindent Unlike human, recognizing and localizing objects still remains a challenge for artificial 
systems, mainly because of two factors - view point dependent object variability and
high in-class variability. In the human detection task intra-class variability includes
color of the clothing, type of clothing, appearance, pose, illumination, partial occlusions
and background. Pedestrian detection has remain most studied problem in the computer vision.
In the last decade with application of CNNs researchers are able to get some nice results 
those are good enough for application in practical scenarios.

\vspace{1em}
\noindent The detection of the human is a partial task where human and machine must interact 
with each other and machine expected to understand human intention. For example, 
a self driving must anticipate and predict the pedestrian in a reliable manner, whether 
the pedestrian intents to cross the road, or standing near the curb or waiting at waiting shelter.
Such intent prediction problem can be viewed as two step related problem. First detect the 
human and track the person in several frames and predict after tracking for certain time using a 
previously trained model for such intention prediction.

\vspace{1em}
\noindent The main idea of this thesis is to take a look at different aspect of solving such a problem, 
starting with reliable data \footnote{There are several image and video datasets exist in the 
public domain for research purpose} acquisition, understanding the annotation in the data, 
training a neural network using efficient features for the detection of person. Detecting the same person in the series of subsequent frames and finally prediction of pedestrian intention using another classifier.

\section{Problem Statement} 
Which model to be used for the first part of the problem, which deals with detecting the pedestrians in the scene? If the model capable of detecting multiple pedestrians in the scene? What is the processing time, if the processing time under a threshold for the purpose of using it in the real time scenario?
Which kind of algorithm to use used for second part of the problem, which deals with predicting the pedestrian intention? What is the time delay of such models if such model can be applicable in real time application scenarios?
How to choose right network for such problems, training them on large dataset which includes several thousand of images or hours of real video recordings? How to set and tune several hyper parameters while training the neural network? This Thesis work mainly answers above question by investigating and  applying several methodologies into practice.

\section{Objectives}
The main idea of this thesis is establishing a Pipeline by selecting appropriate models at both stages and evaluating the end results. Acquiring the public dataset related to pedestrian movement and validating the proposed model on them and estimating the performance in accordance with the dataset owner specification for doing so.

\section{Overview}
The texts in this thesis is split as below.
\begin{itemize}
  \item {\textbf {\textit{Chapter 2}} describes CNN for feature extraction, Deep Learning, state of the art}
  \item {\textbf {\textit{Chapter 3}} explains methodology choosing right algorithm}
	\item {\textbf {\textit{Chapter 4}} presents implementation, and shows the results of various experiments and setups  }
	\item {\textbf {\textit{Chapter 5}} concludes my thesis with a summary of the main results and brief discussion for probable future topics of research}
\end{itemize}

\section{Terminology}
In the current work and by other authors also training and learning is used interchangeably. Let me loosely define some frequently used terms and concepts in the Machine Learning arena.
\\*\textbf{Epoch:}
During the learning the network sees the set of samples several times. During the training, presenting the network entire set of sample once is known as an epoch. So an epoch represents one iteration over entire dataset.
\\*\textbf{Batch and batch size:}
Mainly because of two reasons we can not pass the entire dataset to the network for the training purpose at once. Firstly, datasets by nature most of the time are huge. Let it be image data or some other textual data, in many scenarios it is not possible to feed all the data because of hardware constraints, such as not enough RAM to hold entire dataset.
Secondly to update weight during training process, network has to wait for a very very long time to calculate the delta weight after processing all the input data. To solve this problem usually the full dataset is split into several small batches and number of samples within each small batch is known as 
Batch and batch size.
\\*\textbf{Iterations:} 
Number of batches that a neural network process to complete a single epoch.
The number of  iterations, batch size and  number of data samples in the dataset is given by, the below expression.
\begin{equation}
    D = B * I
\end{equation}
Where D is the total number of samples in the dataset.
\\*B is the number of samples in the mini batch that is fed to the network at once.
\\*I is the  total number of batches the network process in a single epoch.

Larger batch size requires more computational resource and achieves faster completion, in contrast smaller batch size leads to more generalization. In this regards, Yann  LeCun humorously said
\begin{quote}
``Training with large mini batches is bad for your health. More importantly, it's bad for your test error.  Friends, don't let friends use mini batches larger than 32.''
\end{quote}   
The empirical study of the performance of mini-batch stochastic gradient descent in \cite{masters2018revisiting} show that, the team obtained the best training stability and generalization performance using small batch sizes, for a given computational cost, across a wide range of experiments they conducted. In all cases they have achieved the best results with batch sizes m = 32 or smaller.


%optimizations are described in detail in section~\ref{ch1:opts}.

%\section{Description of micro-optimization}\label{ch1:opts}

%multiplies\footnote{Using unnormalized numbers for math is not a new idea; a

%unnormalized arithmetic, with a separate {\tt NORMALIZE} instruction.}.

% This is an example of how you would use tgrind to include an example
% of source code; it is commented out in this template since the code
% example file does not exist.  To use it, you need to remove the '%' on the
% beginning of the line, and insert your own information in the call.
%
%\tagrind[htbp]{code/pmn.s.tex}{Post Multiply Normalization}{opt:pmn}


\subsection{Block Exponent}

In a unoptimized sequence of additions, the sequence of operations is as
follows for each pair of numbers ($m_1$,$e_1$) and ($m_2$,$e_2$).
\begin{enumerate}
  \item Compare $e_1$ and $e_2$.
  \item Shift the mantissa associated with the smaller exponent $|e_1-e_2|$
        places to the right.
  \item Add $m_1$ and $m_2$.
  \item Find the first one in the resulting mantissa.
  \item Shift the resulting mantissa so that normalized
  \item Adjust the exponent accordingly.
\end{enumerate}


\begin{eqnarray*}
a_i & = & a_j + a_k \\
a_i & = & 2a_j + a_k \\
a_i & = & 4a_j + a_k \\
a_i & = & 8a_j + a_k \\
a_i & = & a_j - a_k \\
a_i & = & a_j \ll m \mbox{shift}
\end{eqnarray*}
instead of the multiplication.  For example, to multiply $s$ by 10 and store
the result in $r$, you could use:
\begin{eqnarray*}
r & = & 4s + s\\
r & = & r + r
\end{eqnarray*}
Or by 59:
\begin{eqnarray*}
t & = & 2s + s \\
r & = & 2t + s \\
r & = & 8r + t
\end{eqnarray*}

{\em might\/} 

Challenges are there while just considering image data, 
the verities of image sources for example we can get photos that are taken by 
professionals, synthetic photos drawn by image generators and real life photos 
that we see and capture in our day to day life. So the results of these benchmarks and 
observation across these aforementioned classes of data set does not transfer to the other scenarios.
%% This is an example first chapter.  You should put chapter/appendix that you
%% write into a separate file, and add a line \include{yourfilename} to
%% main.tex, where `yourfilename.tex' is the name of the chapter/appendix file.
%% You can process specific files by typing their names in at the 
%% \files=
%% prompt when you run the file main.tex through LaTeX.
\chapter{State of the Art}
% object detection, pedestrain detection
%\cite{felzenszwalb2009object, walk2010new, liu2016ssd, szegedy2014scalable, dollar2009pedestrian, dollar2011pedestrian}

% feature extraction
%\cite{dalal2005histograms, lowe2004distinctive}

% Region Proposal
%\cite{girshick2014rich, girshick2015fast, ren2015faster}

% Trackers
%\cite{fiaz2019handcrafted,xu2005pedestrian,ning2017spatially}

% Activity prediction
%\cite{ryoo2011human, richard2017weakly }


% Intention prediction
%\cite{saleh2017intent, fang2018pedestrian, abu2018will, dollar2005behavior, cao2017realtime, rasouli2017agreeing }

% Trajectory prediction
%\cite{saleh2017intent, zhang2019sr, xue2018ss, lipton2015critical}

The problem of human action classification or pedestrian intent prediction, pedestrian future trajectory estimation \cite{saleh2017intent, fang2018pedestrian, abu2018will, dollar2005behavior, rasouli2017agreeing, cao2017realtime} from a sequence of images has got huge attention from the researchers in recent years. This problem is inherently complex as it is based on other complex problems. To predict future location, detect the action or intention one must first detect the object in an image and classify it, track the same object over several continuous frames. Before the dominance of the convolutional neural network (CNN) and Deep Learning which has outperformed other traditional methods in recent times. For any such prediction task, we need an efficient and robust object detection, object localization and object tracking model working seamlessly together to enable the last phase of the prediction task. Object detection and localization constitute the backbone of this pipeline and takes maximum share of the run time in the entire pipeline. To build a real-time pedestrian future bounding box prediction pipeline, it is of utmost important to find a model that performs real-time object detection \cite{felzenszwalb2009object, walk2010new, liu2016ssd, szegedy2014scalable, dollar2009pedestrian, dollar2011pedestrian} and localization reliably. In this chapter, some of the state-of-the-art models for such tasks are explored.

\section{Pedestrian intention on roads}
\newpara Abu Farha and the team presented \cite{abu2018will}, where they discussed methods to predict actions that are in a considerable future and their duration. And they tried to anticipate all activities within a horizon of 5 minutes. After inferring the activities from the observed part of the video using an RNN-HMM \cite{richard2017weakly}, they proposed two approaches to predict the future actions and their durations, in the first approach they used RNN, where the anticipated activities are fed to the RNN to predict the remaining duration of the ongoing activities and duration of the next activity along with its class.
In the second approach, the proposed CNN based single-pass model which predicts the length and label of the future action. They have found that both approaches have outperformed both the grammar-based baseline and the nearest neighbor baseline. RNN and CNN found to be performing similarly for time horizon more than 40 seconds and RNN performs better for lesser than 20 seconds time horizon. The task is more formally defined as,
given the first t frames $X_{\text{1}}^t$ \\
predict \[ C_{t+1}^T  = \ (C_{t+1}, ..., C_{T}) \]
where $C_{\text{i}}$ denotes action labels for unobserved frames
and the video is given by
\[ X_{1}^T = (X_{1}, ..., X_{T}) \]

\newpara
For the RNN training, the loss function used as below
\begin{equation}
    L = -log\, \hat{p_c} + (l_r - \hat{l_r})^2 +  I (l_n - \hat{l_n})^2 
\end{equation}
where \\
$\hat{l_r}$ represents predicted remaining length of the current action in seconds, \\
$\hat{l_n}$ represents the predicted length of the next action in seconds, \\
$\hat{p_c}$ is the predicted class of the next action

\newpara Unlike the recursive strategy in RNN, the CNN approach does the prediction in a single pass.
Training data generated by using the first 10\%, 20\%, 30\%, 50\% of the video as the observation and the following 50\% as the ground truth. During the training of the network squared error used as loss function.
\begin{equation}
    L = \frac{1} {SC} \sum_{a,c} (Y_{sc} - \hat{Y_{sc}})^2 
\end{equation}
where $\hat{Y}$ is the prediction of the network. \\
C represents action classes and \\
S number of rows for an action segment

\newpara Row-wise $\textit{l}_2$-normalization \footnote{$l_2$-norm of a real vector $x=(x_1,x_2,x_3)$ is given by $|x|=sqrt(x_1^2+x_2^2+x_3^2)$} of the output found to be more robust than softmax output with cross-entropy loss. %\cite{abu2018will} did not elaborate the 'input sequence always end 1 second before the next action segment start'. What is the purpose of this and results if the input sequence is provided till the next action segment starts.

For the task of early detection, the goal is to recognize the activity with the least possible amount of input observation \cite{ryoo2011human}. In \cite{ryoo2011human} they modeled the feature distribution over the course of observation by integral histogram representation of activities and named the prediction algorithm as \textit{dynamic bag-of-words} as the prediction algorithm considers the sequential structure formed by video features. And they formulated the activity prediction process probabilistically as:
%\hat{f}(x,y) = \underset{(s,t)\in S_{xy}}{\mathrm{median}} \{g(s,t)\}
%\begin{equation}
%\begin{multlined}
\begin{align}
\begin{split}
	%\displaystyle \sum_{n=1}^\infty
		P(A_{p}\: |\: O,t) ={}& \displaystyle \sum_{d}  P(A_{p},d\: |\: O,t)\\
		={}&	\frac{\sum_{d} P(O\: | \: A_{p},d)P(t\: | \:d)P(A_{p},\: d)}
	 {\sum_{i}\sum_{d} P(O\: | \: A_{i},d)P(t\: | \:d)P(A_{i},\: d) }
\end{split}
\end{align}
%\end{multlined}
%\end{equation}

\newpara Where \textit{d} is a progress level of the activity \\
\textit{$A_{p}$}  \textit{ P(t|d)} represents the similarity between the observation length \textit{t} and that of the activity progress \textit{d}
\textit{P(O|Ap, d)} measures the similarity between the video observation and the activity \textit{$A_{p}$} with progress level \textit{d}. This \textit{integral bag-of-words} uses 3-D space-time local features and detect motion changes in the video and generates descriptors representing local movements in the video. These are cuboid feature descriptors \cite{dollar2005behavior}. After the features are extracted, \textit{k-means} clustering was applied and the clusters are known as visual words. Every feature detected belongs to one of these k-visual words. Then the activities are model as an integral histogram of visual-words. The similarity between a video and an activity model was done by comparing their histogram representation. And finally, dynamic programming was used to predict the ongoing activities from videos.

\section{Prediction (estimation) methods }
The last stage of the pipeline predicts the future location of the visual object \cite{saleh2017intent, zhang2019sr, xue2018ss, lipton2015critical}. These algorithms predict the bounding box for a tracked pedestrian or predict the intention of the pedestrian, whether the pedestrian wants to cross in the near future or not. Subsequent section shall discuss some of the state-of-the-art methodology that addresses this.
In \cite{saleh2017intent} Saleh et.al presented a data-driven approach which is different from the older approach which employs dynamical motion modeling and motion planning. They have formulated the intent prediction problem as a time-series problem. They proposed the method by observing a short window sequence of the pedestrian motion trajectory, a prediction about their future latent position can be done up to 4 secs ahead. With a deep-stacked LSTM network, they achieved competent results on the Daimler testing data set. In their work, three LSTM layers are stacked. The input layer takes a 2-dimension sequence with a window size 10 of the lateral position of pedestrian motion trajectory. Last LSTM layer connected to a fully connected layer that has only one neuron and predicts lateral position at the next time step. A linear activation function is used in the fully connected layer. In this method, only the next time step value is predicted. At the inference time, recursive prediction is used to predict any window-sized sequence. Prediction with different size input sequences is achieved by this.
As part of training this LSTM  model, which is an optimization problem, Mean Squared Error (MSE) is used as a loss function.

\begin{equation}
MSE= \frac{1}{N}\sum_{i=1}^{N}(\hat{Y_i} - Y_i)^2
\end{equation}

They presented their work on the Daimler pedestrian path prediction benchmark dataset, which consists of 68 stereo image sequences. This sequence includes four scenarios 'Crossing', 'Stopping', 'Starting' and 'Bending In'. These sequences are annotated with frame-wise pedestrian bounding boxes, a median disparity of the upper body area of the pedestrian, the position of the pedestrian in the vehicle coordinate system and time to event information. This method achieved superior results in both short and long term predictions for the four scenarios with the test data.
In \cite{zhang2019sr}, Zhang et.al presented a work, which includes the current intention of the neighbors and iteratively refines the current states of all participants in the crowd using a message passing mechanism. Social behaviors are stressed upon and learned using LSTM, but \cite{zhang2019sr} neglected factors such as current states of neighbors and adapting selected information from neighbors based on their motions and locations. In their work, they introduced a State Refined module as a subnetwork of the LSTM cells. This subnetwork aligns pedestrians together and updates their current state. Formulation for the cell state was given by 
\begin{equation}
\hat{C}^{t, l+1}= \sum_{j\in N(i)}M(\hat{h_j}^{t, l}, {h}^{t, l}) + \hat{C}^{t, l}
\end{equation}
Where M is the message passing function and used to calculate social information from the neighboring pedestrian. They also used \textbf{Pedestrian-wise }attention and \textbf{Motion gate} to select important information from neighboring pedestrian for message passing. However, in this paper, they have not described its usefulness in the vehicle and Pedestrian context. This motivated me to explore vehicle and pedestrian social awareness and use that information.


\cite{alahi2016social, xue2018ss} discuss the social aspect of pedestrians in the scene. In \cite{xue2018ss}, Xue et. al used a hierarchical LSTM model for Pedestrian Trajectory Prediction. They have considered the influence of social neighborhood and scene layouts. In their model, they used three different LSTMs to capture person, social and scene scale information in a circular shape neighborhood. The dataset and scenarios described in this paper are humans in a crowded place. And it has not explored whether pedestrian crossing or trying to cross in a traffic scenario. However some of the findings still applicable in the pedestrian-moving vehicle scenario as well.


\section{Features extraction methods}
Object detection \cite{felzenszwalb2009object, walk2010new, liu2016ssd, szegedy2014scalable, dollar2009pedestrian, dollar2011pedestrian} is the core for pose estimation and object tracking. The accuracy of this stage very important for other stages to perform. This detection stage identifies the object and locates it. In this section, various new algorithms have discussed those deals with object detection.

\subsection{Manual feature extraction}
Pedro F. Felzenszwalb and the team proposed a deformable part model (DPM) \cite{felzenszwalb2009object} which is based on pictorial structures that represent objects by means of a collection of object parts arranged in a deformable configuration. Each object part captures the local appearance properties of
an object and the deformable configuration is identified by spring-like connections between certain pairs of such parts. They use a variation of Support Vector Machine(SVM) which they named as latent SVM (LSVM) is used for the training of the model and  histogram of oriented gradients (HOG) used for features vector. After the success of usages of CNNs in tasks like image classification, CNNs are used to extract the feature sets instead of manual extraction.

\newpara \textbf{HOG }
In the year 2005 Dalal \& Triggs found a method that is superior to Haar wavelets descriptor and other state-of-the-art method based on edge and gradient for the task of pedestrian detection. HOG descriptors are comparable with edge orientation histogram, SIFT \footnote{Scale Invariant Feature Transform, by transforming image data into scale-invariant coordinates relative to local features \cite{lowe2004distinctive}} descriptors and shape context, however, these are computed on a dense grid of uniformly spaced cells and use overlapping local contrast normalization. Which they named as Histograms of Oriented Gradient (HOG) features\cite{dalal2005histograms}. The gradient is computed by applying a filter kernel \\
\begin{center}
$[-1,0,1] \, and \, [-1,0,1] ^{T}$
\end{center}

\newpara After computation of the gradient, their values are used to update a 9 histogram channel which is evenly spread over 0-180 degrees or 0-360 degrees. They tried person/non-person classification using this robust feature descriptor and Linear SVM and got state-of-the-art results of that time.

\subsection{CNN based feature extraction}
\newpara CNNs found to outperform manual feature extraction as they are principally designed Neural net \textbf{architecture which preserves local connections and shares weights}. CNN is a special type of multilayer perceptrons, based on shared weight architecture and inspired by biological connectivity between neurons in the human visual cortex. A particular neuron responds to stimuli only in a restricted region of the visual field called receptive field. CNN employs a special kind of linear operation called convolution. CNN is characterized by a series of several convolutions, non-linearity, pooling operation and finally followed by one or more fully connected layers. Initial convolutional layers are used as feature extraction and final layers as fully connected. The fully connected layer takes the input from the convolutional network and produces an N-dimensional vector, where N is the number of classes the model is intended to select.


\section{Region of Interest proposing methods}
Object localization is one step forward when compared to object recognition. Identifying whether an object is present in the image or not is the task of object recognition, however, object localization deals with locating where the object is located in an image. There are several approaches used for localization as discussed below. And it is complex in comparison to image classification because a huge number of candidate object locations must be processed during the localization process. \\

\textbf{Sliding Window:} A classifier is trained on the objects first and in the next phase a window is moved over the whole image at different scales. Each window gets a score from the learned classifier and window(s) with the highest score predicts the location of an object. Sliding Window technique is computationally inefficient. Some of the improved and modern methods are presented below. 

\subsection{R-CNN Introduction}
In \cite{girshick2014rich} Girshick et al. combined region proposals with CNN and named the method as R-CNN. This method generates around 2000 category independent region proposals for the input image, using a CNN, a fixed-length feature vector is extracted regardless of the region shape. Then each region is classified using class specific linear SVM. 

\newpara \textbf{R-CNN:} \\
The features for a region proposal \cite{girshick2014rich, girshick2015fast, ren2015faster} are computed via a CNN proposed by Krizhevsky et al. This network takes a mean-subtracted 227 x 227 RGB image as input. Therefore image data in the region proposal is converted to 227 x 227 as required by the Krizhevsky network. For this, an arbitrary-shaped  proposed region is dilated and warped. During the test time using selective search fast mode 2000 region proposals are generated and warped and processed through CNN to extract desired features. After that, the per-class basis score is evaluated using the SVM trained for that class. Subsequently, a greedy non-max suppression for each class is applied independently which rejects region that has IoU overlap with higher scoring selected region larger than a threshold.

\newpara \textbf{Fast R-CNN:} \\
In \cite{girshick2015fast} an improved technique is discussed which is based on previously discussed R-CNN. This is computationally faster than R-CNN and results in faster training and testing speed. R-CNN training is a multi-stage pipeline. It first fine-tunes a ConvNet on object proposal and then fits SVMs to ConvNet features. Because of feature extraction from each object proposal, without sharing computation in each test image, object detection is relatively slow in R-CNN. Based on the concept described in SPPnets, a convolutional feature map is generated for the entire input image and subsequent classification is done by extracting features from the shared feature map for each object proposal. Features for the object proposal are extracted by max-pooling the portion of the feature map inside the  object proposal region into a fixed size output. During the training, Fast R-CNN takes an entire image and set of object proposals, for each object proposal an RoI layer extracts a fixed-length feature vector from the feature map. Each feature vector is further fed into fully connected layers which produces a feature vector that is branched into two output layers, one outputs softmax probability over K object classes and other layer produces four real numbers for each of the K object classes termed as \textit{bbox regressor}. The architecture is trained with multi-task loss function via backpropagation. Fast R-CNN uses a softmax classifier instead of a linear SVM that was used in R-CNN.

\newpara \textbf{Faster R-CNN:} \\  
Another enhanced proposal came from the Author of previously discussed R-CNN methods with a goal of sharing computation with a Fast R-CNN for the task of region proposal. In \cite{ren2015faster}, a new Region Proposal Network (RPN) is introduced that shares full-image convolutional features with the detection network, thus almost eliminating the cost for the region proposals. RPN is a fully convolutional network that simultaneously predicts object bounds and objectness scores at each position. This RPN is trained to generate high-quality region proposals, which are used by Fast R-CNN for detection. By sharing convolutional layers with the object detection network, RPN has a marginal cost for computing proposal \footnote{in the order of 10 ms per image}. For the very deep VGG-16 model this methodology achieved 5fps on GPU and accuracy of 73.2\%mAP for PASCAL VOC 2007 using 300 proposals per image. 

\subsection{Model-based on R-CNN}
\newpara Estimating pedestrian future using pedestrian dynamic model were done in the past and those models were difficult to adjust and achieve robustness, they required high-quality stereo data, dense optical flow and ego-motion compensation which required vehicle data.  Using the 2D pose estimation method, that is applied to the still images in the sliding window manner, a state-of-the-art result has been obtained\cite{fang2018pedestrian} for the C/NC task with the Daimler dataset. In the pipeline of their task, they used the below components. They have used generic off-the-shelf CNN modules as Detector, Tracker and Pose Estimator in their pipeline.

\begin{itemize}
	\item Detection: Fine-tuned Faster R-CNN \cite{ren2015faster} based on VGG16 CNN architecture. 
	\item Tracking: Object tracking-by-detection \cite{wojke2017simple}, purely image-based off-the-shelf solution 
	\item Pose Estimation: CNN-based pose estimation method discussed in \cite{cao2017realtime}
	\item Prediction: Random Forest based on 4096T dimensional vector, where T is the number of frames tracked
\end{itemize}

\newpara
Researcher Fang and team in their 2018 paper \cite{fang2018pedestrian} described image-based 2D pose estimation for detecting pedestrian intention: whether the pedestrian crossing the road, stopping before entering the road, starting to walk or bending towards the road. They performed their experiment with the choreographed Daimler dataset. They also used publicly available, a rather new dataset (JAAD) \cite{kotseruba2016joint} that allows the development of methods and experiments in a more naturalistic driving condition. They used CNN based pedestrian detection, tracking and pose estimation to predict the crossing action from monocular images. In their paper, they have mentioned that without additional information such as stereo, optical-flow or ego-motion compensation, they achieved state-of-the-art results with only image-based 2D pose estimation. The prediction of the action is based on per pedestrian multi-frame feature set extracted using last \textit{k} frames. Estimating the pose was central to the prediction of the crossing intention. In the results, they have observed that classification with respect to features based on the skeleton of the pedestrian outperformed the features based on CNN's fc6 layer.

\subsection{Single Shot Multi-Box}
There are several proposals proposed for person detection, many of them try to achieve high accuracy at the expense of high computational cost. This leads to  many such proposals not useful for real-time applications such as driverless cars. Single Shot Multi-Box methodology aims at faster detection and suitable for real-time applications.  
SSD motivated by the ideas presented in \cite{szegedy2014scalable} where a convolutional network is trained to output the coordinates of the object bounding boxes. MultiBox loss is the weighted sum of \textit{confidence} loss and \textit{location} loss. Single Shot Multi Box with aforementioned inspiration attacked the problem with several small improvements and variations, achieving state-of-the-art results with still simpler architecture, in comparison to \cite{szegedy2014scalable} which uses a separate deep neural network to generate a proposal for the bounding box.

\subsubsection{SSD}
Faster R-CNN operates at only 7 frames per second (FPS) SSD aimed at improving the speed by employing some new methods which does not re-samples pixels or features for the bounding hypothesized boxes. This improves the speed substantially without much or no decrease in accuracy. SSD is implemented and used with the Caltech dataset \cite{dollar2009pedestrian} during the thesis period and details to be discussed in the subsequent chapters.

\newpara The central principle on which SSD \cite{liu2016ssd} is based on is, discretizing the output space of bounding boxes into a set of pre-determined default boxes consisting of different aspect ratios and scale per feature map location. The principle used as part of SSD overcomes the challenge faced in \cite{ren2015faster} and its predecessors which includes multi-phase training and slow inference time. SSD does this by doing both the task of object localization and classification in a single forward pass. It uses a \textit{MultiBox} approach, which pre-computes priors \footnote{Alternatively known as anchors in Faster R-CNN}, (fixed size bounding boxes). In MultiBox based approach, prediction starts with priors and try to regress closer to the ground truth bounding boxes.
It uses a combination of confidence loss and location loss for calculating the loss of the model. Confidence loss uses categorical cross-entropy \footnote{otherwise known as softmax loss} and location loss uses Smooth L1-Norm \footnote{\textit{L1}-norm is well known as \textit{Manhattan }norm}.

\begin{equation}
	\left \| x_1 \right \| =\sum_i(x_{1_i})
\end {equation}

L1-norm gives the distance between two vectors, known as \textit{Sum of Absolute Difference} distance given by the below equation:
\begin{equation}
	SAD(x_1,x_2) = \left \| x_1-x_2 \right \|_1 = \sum \left | x_{1_i}-x_{2_i} \right |
\end {equation}

\textbf{Fixed Priors:} In the case of MultiBox, the priors are chosen according to their IoU with respect to ground truth above a certain defined ratio. \footnote{Usually 0.5 is considered as threshold value}.
However, in the case of Fixed Priors, carefully, a set of different sizes and aspect ratio priors are chosen manually. This lead to the removal of the pre-training phase for the prior generation. For a feature map with \textit{b} default bounding boxes per cell and model is trained for \textit{c} classes, then the number of values for the feature map $(f)$ is given by the relation 
\begin{equation}
	f = (m * n )  (4 + c) * b
\end {equation}
4 in the above equation indicates a number of co-ordinate values for two corner points or 4 offset relative to the original default box shape. \\
m x n is the shape of the feature map \\
 An additional point to note is, the more the number of default boxes, the more accurate is the detection at the cost of speed. Predicting category scores for multiple classes and box offsets for pre-computed bounding boxes using small convolutional filters applied to feature maps lead to a faster single-pass object classification and localization system. And prediction of different scales from feature maps of different scales and separate prediction by aspect ratio leads to higher accuracy. \cite{liu2016ssd} With the same VGG-16 base architecture SSD300 model runs at 59 FPS. The mentioned result from SSD paper is very interesting and suitable for the application of recurrent neural networks to detect and track the object in video simultaneously in real-time.

\subsubsection{YOLO}
Another variation of single-shot detection during the same time achieving the state-of-the-art result was YOLO. Single network-based detection pipeline. YOLO divides the image into a grid of cells and predicts the coordinates and confidences of objects contained in the cells. Authors like Szegedy et.al in \cite{szegedy2014scalable} have expressed uncertainty about its performance in the situation where there are significantly more objects in a data set. As SSD seems to performs better with respect to run time and performance, YOLO was not studied further during the thesis period.

\section{Tracking methods}
Visual object tracking is also a very active area of research, several algorithms are discussed in \cite{fiaz2019handcrafted,xu2005pedestrian,ning2017spatially}.
In \cite{fiaz2019handcrafted} Fiaz et. al broadly categorized trackers into Correlation Filter based Trackers (CFTs) and Non-CFTs. Human trackers can be divided into motion models or appearance models. They experimented with 24 recent trackers and found that trackers using deep features performed better than handcrafted. In some cases, the fusion of both increases performance significantly. Based on their study they have concluded that Discriminative Correlation Filter (DCF) based trackers perform better than others. There are some notable implementation e.g. 
Kalman Filter with Mean shift tracking, Recurrent YOLO based spatially supervised RCNN \cite{ning2017spatially}, a deep neural network that uses raw video frames as input and its output is the coordinates of a bounding box of an object being tracked in each frame. Single night vision camera can also be used in the night for tracking of the pedestrian with help of Kalman filter and mean shift. Some of sensor fusion techniques improves the tracking as well as detection result when IR sensor, RADAR, laser output are processed together with the camera image. Fang et.al discussed in their survey paper In \cite{gonzalez2016pedestrian}, a combination of visible and non-visible imaging techniques increasing detection accuracy. More in-depth research is out of the scope of the thesis topic, so no further study is done in this area during the thesis period.




\appendix
\chapter{Tables}

\begin{table}
\caption{Armadillos}
\label{arm:table}
\begin{center}
\begin{tabular}{||l|l||}\hline
Armadillos & are \\\hline
our	   & friends \\\hline
\end{tabular}
\end{center}
\end{table}

\clearpage
\newpage

\chapter{Figures}

\vspace*{-3in}

\begin{figure}
\vspace{2.4in}
\caption{Armadillo slaying lawyer.}
\label{arm:fig1}
\end{figure}
\clearpage
\newpage

\begin{figure}
\vspace{2.4in}
\caption{Armadillo eradicating national debt.}
\label{arm:fig2}
\end{figure}
\clearpage
\newpage

%% This defines the bibliography file (main.bib) and the bibliography style.
%% If you want to create a bibliography file by hand, change the contents of
%% this file to a `thebibliography' environment.  For more information 
%% see section 4.3 of the LaTeX manual.
\begin{singlespace}
\bibliography{main}
\bibliographystyle{plain}
\end{singlespace}

\end{document}

