%% This is an example first chapter.  You should put chapter/appendix that you
%% write into a separate file, and add a line \include{yourfilename} to
%% main.tex, where `yourfilename.tex' is the name of the chapter/appendix file.
%% You can process specific files by typing their names in at the 
%% \files=
%% prompt when you run the file main.tex through LaTeX.

\chapter{Conclusion and Future Work}
In the presented master thesis, i have looked various aspects of pedestrian behavior (pedestrian intention, pedestrian future location). The research is split in three main phases. In the first phase, end to end pipe line for pedestrian detection, tracking and future bounding box is understood. In the second phase, an experiment for pedestrian detection using Caltech dataset and an open implementation of SSD is carried out. With a small scale SSD network and resource constrained environment, it is found that SSD can be used in the above mentioned pipeline where real time detection is of utmost importance. It is observed with the trained SSD model to be performing at 10fps with mAP(mean Average Precision) of 66 percentage. The third phase involves pedestrian bounding box prediction, a Deep Learning  based method based on LSTM (a variation of RNN) is used with a recently published JAAD-pedestrian dataset. After the LSTM model is trained and validated, several experiments are conducted, and the results are presented. \\

From the experiments conducted, whose details are presented in the last chapter, with 15 frames of input, prediction at 15 future frames is better than prediction at 30 or 60 frames. With the experiment where predictions are made at various different future frame with 60 frames as input, performance for prediction at 30frames outperforms the other predictions at 60 frames and 90 frames. Thus it is concluded that the prediction accuracy decreases with increase in numbers of future frames. As seen in the graph , the accuracy drops exponentially. It is also observed that mean center distance also increases linearly with increase in prediction at particular future frame keeping number of input frames fixed. \\

It is also concluded that, with increase in LSTM cell memory capacity the prediction increased initially and after that it started to drop. In our case, the model is found to be performing best with 100 units. \\

From the prediction run time graph, it is concluded that, with the increase in number of input frame the prediction time increases linearly. Impact due to change in number of epochs or amount of unit in the cell is insignificant.

The LSTM Cell state refinement proposal is made and related mathematical aspect are reviewed and presented in the section \ref{state_refinement}

It is estimated that best prediction at 30 frames is done with 60 frames as input. So it can be concluded that for prediction at particular future time slice, a certain number of prior frames as input performs better than more number of input frames(than desired).

\newpara

\textbf{Future Work}
For the future work, further investigation may be done on how resulted proposal can be applied to real-world scenarios, because following concerns are still open: \\
for the implementation of the SR (state refinement \ref{state_refinement}) module, $\lambda$ should be pre-computed \ref{sf-precompute}. In the proposal for the state refinement, the distance from the camera to the least point which is captured in the image space is considered to be a constant \textit{K}, that constant value can be empirically evaluated.

This work can be extended with focusing other useful features in the JAAD dataset such as pedestrian behavioral information to improve the accuracy and precision..

A different approach can also be followed for choosing the accuracy evaluation criteria for the pedestrian bounding box by combining both the IoU and mean center distance.
%\section{Old Text}
%In general we can think of two types activity prediction, (i) early activity detection (ii) future activity prediction. In the case of early activity detection, the class label of an action is inferred at the point when the activity starts or shortly after the activity started. Where as in the future activity prediction, the class label of the action that will happen next is predicted and also the starting time in the future is also predicted.